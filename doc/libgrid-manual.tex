\documentclass[12pt,letterpaper]{report}
\usepackage[letterpaper,hcentering,vcentering,left=1in,top=2.45cm,right=1in,bott
om=2.45cm]{geometry}
\usepackage[latin1]{inputenc}
\usepackage{url}
\usepackage{longtable}
\usepackage{amsmath}
\usepackage{amsfonts}
\usepackage{amssymb}
\usepackage{graphicx}
\author{Jussi Eloranta}
\title{libgrid manual}
\begin{document}

\maketitle

\chapter{Prerequisites}

\section{Introduction}

Libgrid provides routines for efficiently accessing and manipulating 1-D, 2-D, and 3-D Cartesian real and complex grids on unix-based systems without the need of considering the underlying specialized hardware (e.g., OpenMP, CUDA). In addition to the basic grid operations such as grid allocation, release, Fourier transform, grid arithmetics, etc., it has also specialized routines for propagating time-dependent Schr\"odinger equation (linear or non-linear) in real or imaginary time. The latter routines are required, for example, by libdft library, which solves various types of non-linear Schr\"odinger equations that describe superfluid $^4$He (Orsay-Trento) and Bose-Einstein condensates (Gross-Pitaevskii). Libgrid was written by Jussi ELoranta with contributions from Lauri Lehtovaara and David Mateo. It can be freely distributed according to GNU GENERAL PUBLIC LICENSE Version 3 (see doc/GPL.txt). This project was partially supported by National Science Foundation grants: CHE-0949057, CHE-1262306 and DMR-1205734.

\section{Installation}

Installation of libgrid requires the following packages:
\begin{itemize}
\item git (open source distributed version control system)
\item GNU C compiler with OpenMP support (gcc)
\item FFTW 3.x (Fast Fourier Transform package)
\item Grace 2-D plotting program (xmgrace) and its development libraries
\item NETCDF library for converting to CDF format
\end{itemize}
To install these packages on Fedora linux, use (\# prompt implies execution with root privileges): 
\begin{verbatim}
# dnf install git gcc fftw-* grace grace-devel netcdf netcdf-devel
\end{verbatim}
or
\begin{verbatim}
# apt install git gcc fftw3-dev grace libnetcdf-dev
\end{verbatim}
If the system has NVIDIA GPUs, libgrid can use CUDA to accelerate the grid operations. For Fedora-based systems, the propriatary NVIDIA driver (i.e., apt install nvidia-driver) and the CUDA libraries can be obtained from: \url{https://negativo17.org/nvidia-driver/}. For Debian enable the testing repository in /etc/apt/sources.list so that you can install the latest version of NVIDIA drivers and CUDA. Both systems are compatible with the default settings in make.conf as described below. Currently, libgrid does not support AMD based GPUs. libgrid uses the following CUDA packages: CUFFT, CURAND, and the basic CUDA runtime (libcuda and libcudart). Also, the CUDA compiler (nvcc) must be installed to compile GPU code. So, on Debian you should do (there is a similar dnf install for Fedora):

\begin{verbatim}
# apt install nvidia-cuda-dev nvidia-cuda-gdb nvidia-cuda-mps 
# apt install nvidia-cuda-toolkit nvidia-cuda-toolkit-gcc
\end{verbatim}

To copy the current version of libgrid to subdirectory libgrid, issue (\% implies execution with normal user privileges):
\begin{verbatim}
% git clone https://github.com/jmeloranta/libgrid.git
\end{verbatim}
Change to libgrid directory and review the configuration settings:
\begin{verbatim}
% cd libgrid
% more make.conf
\end{verbatim}
The options specified in this file determine how libgrid will be compiled. OpenMP support is included automatically with COMPILER = gcc. Invoking debugging will disable OpenMP and the related \#pragma's in the code will show a bunch of warnings when compiling (this is normal). Intel or Portland compilers have not been tested with libgrid. The configuration options are described in the table below.\\

\begin{longtable}{p{.43\textwidth} p{.55\textwidth}}
\textbf{Option} & \textbf{Description}\\
\cline{1-2}
COMPILER & gcc (normal use) or gcc-debug (non-parallel debugging version)\\
REAL & Floating point number precision (float, double or quad precision). Note that for many applications single precision tends to have very limited use.\\
INT & Integer number size (int or long)\\
ROOT & Root installation directory (default /usr)\\
CUDA & Set to "yes" if CUDA is installed and "no" if not. The default is to auto-detect, so changing this is not usually required.\\
CUDAINC & Include file directory for CUDA C header files (default /usr/include/cuda; Fedora). Will also search /usr/include (Debian).\\
CUDALIB & Directory containing the CUDA libraries (default /usr/lib64).\\
CUDA\_HOSTCC & GCC version that is compatible with the current installation of CUDA (default is to auto detect; works on Fedora and Debian).\\
CUDA\_ARCH & NVIDIA GPU architecture (e.g., sm\_50; default auto-detect).\\
CUDA\_FASTMATH & "yes" adds NVCC options: --ftz=true --prec-div=false --prec-sqrt=false --fmad=true and "no" implies options: --ftz=false --prec-div=true --prec-sqrt=true --fmad=true (default no). Seems to provide no performance improvement.\\
CUDA\_TPB & Number of CUDA threads per block (GPU architecture dependent; default 8).\\
CUDA\_THRADJ & CUDA Thread adjustment for 2-D operations (i.e., Crank-Nicolson). GPU architecture dependent value, 2 - 3 (default 3).\\
CUDA\_DEBUG & Whether to include debug code in libggrid CUDA routines ("yes" or "no"; default no).\\
\end{longtable}

\noindent
To compile the library, change to src subdirectory and issue make (-j does parallel compilation):
\begin{verbatim}
% cd src
% make -j
\end{verbatim}
Provided that the compilation completed without errors, install the library (as root unless you changed the ROOT setting above to point to your own directory):
\begin{verbatim}
# make install
\end{verbatim}
To compile the utilities required for file format conversions and interface with graphical applications, compile and install the utilities:
\begin{verbatim}
% cd ../util
% make
\end{verbatim}
Once the utilities compile, install them (as root unless you changed ROOT above to point to your own directory):
\begin{verbatim}
# make install
\end{verbatim}

\noindent
The recommended OpenMP environment variable settings for optimal performance are (csh/tcsh syntax):
\begin{verbatim}
setenv OMP_SCHEDULE auto
setenv OMP_PROC_BIND TRUE
setenv OMP_WAIT_POLICY ACTIVE
\end{verbatim}
Depending on your login shell, these can be added to your .tcshrc, .cshrc, .profile, or .bashrc file. For the latter, the commands are (sh derivatives such as bash):
\begin{verbatim}
export OMP_SCHEDULE=auto
export OMP_PROC_BIND=TRUE
export OMP_WAIT_POLICY=ACTIVE
\end{verbatim}

\chapter{Programming interface}

\section{Accessing the library routines}

To access libgrid functions in C program, the following header files should be included:
\begin{verbatim}
#include <grid/grid.h>
#include <grid/au.h>
\end{verbatim}
where the first include is required to access libgrid routines and the second (optional) include enables the following conversion factors:\\

\begin{longtable}{ll}
GRID\_AUTOANG & Factor to convert from Bohr to \AA{}nstr\"om ($10^{-10}$ m).\\
GRID\_AUTOM & Factor to convert from Bohr to meter.\\
GRID\_AUTOK & Factor to convert from Hartee to Kelvin (energy; $kT$).\\
GRID\_AUTOCM1 & Factor to convert from Hartree to wavenumber.\\
GRID\_HZTOCM1 & Factor to convert from Hz to wavenumber.\\
GRID\_AUTOAMU & Atomic unit mass (electron mass) to atomic mass\\
              & unit (AMU).\\
GRID\_AUTOFS & Atomic unit time to femtosecond.\\
GRID\_AUTOS  & Atomic unit time to second.\\
GRID\_AUTOBAR & Atomic pressure unit (Hartree/Bohr$^2$) to bar.\\
GRID\_AUTOPA & Atomic pressure unit to Pa (Pascal).\\
GRID\_AUTOATM & Atomic pressure unit to atm.\\
GRID\_AUTOMPS & Atomic velocity unit to m/s.\\
GRID\_AUTON & Atomic force unit to Newton (N).\\
GRID\_AUTOVPM & Atomic electric field strength to V/m.\\
GRID\_AUTOPAS & Atomic viscosity unit to Pa s.\\
GRID\_AUKB & Boltzmann constant in a.u. ($k_B$)\\ 
\end{longtable}
\noindent
Note that librid will use the atomic unit system and the above conversion factors are provided for converting to other unit systems. To convert from atomic unit to another unit, multiply by the predefined factor above or, divide by it in order to convert the other way around.

To compile and link a program using libgrid, it is most convenient to construct a makefile (note that the \$(CC) line has TAB as the first character):
\begin{verbatim}
include /usr/include/grid/make.conf

test: test.o
    $(CC) $(CFLAGS) -o test test.o $(LDFLAGS)

test.o: test.c
\end{verbatim}
This will compile the program specified in test.c and link the appropriate libraries automatically. Both CFLAGS and LDFLAGS are obtained automatically from libgrid's make.conf (the first line). If you used a different installation ROOT setting, replace /usr above with that.

\section{Data types}

The libgrid header file defines the real and integer data types automatically according to those requested during the configuration step (make.conf). Instead of using float, double, int, long etc. directly, type REAL to represent a floating point number and INT for integer (e.g., REAL x, y; INT i;). To define a complex number, use REAL complex. Since the function names in the system math library vary depending on the floating point precision (e.g., expf(), exp(), expl()), libgrid defines the corresponding functions in upper case such that is assigned to the function of the requested precision (e.g., EXP(), SIN()). So, in order to call the exponential function, use, e.g., y = EXP(x);. Another place where the size of the REAL and INT data types are required is the scanf/printf standard I/O library routines. To print a REAL number with printf, use, for example, printf("value = " FMT\_R "$\backslash$n");. Or to print an INT use FMT\_I instead. Inside the library itself, EXPORT keyword is added in front of every function that should be visible to applications (automatic generation of prototypes).

Libgrid has built-in data types for 1-D, 2-D, and 3-D Cartesian REAL ("rgrid") and REAL complex ("cgrid") grids. To allocate such grids in a C program, first introduce them as pointers, e.g., rgrid *abc or cgrid *abc. Then use either rgrid\_alloc() or cgrid\_alloc() functions to allocate the space. For example:
\begin{verbatim}
   cgrid *abc;
   abc = cgrid_alloc(32, 32, 32, 1.0, CGRID3D_PERIODIC_BOUNDARY, NULL);
   ...
   cgrid_free(abc);
\end{verbatim}
would allocate a periodic 3-D grid with dimensions 32x32x32 and spatial grid step length of 1.0 (for more information on the arguments to cgrid\_alloc(), see Section 5). To request a 1-D grid, include its dimension last, e.g., 1x1x32, or for a 2-D grid use the last two dimensions, e.g., 1x32x32. It is important to follow this convention for performance reasons. In the code, coordinate $x$ corresponds to the first index, $y$ to the second, and $z$ to the third. By default, the origin is placed to the center of the grid.

The basic properties of the rgrid and cgrid data types are explained below.

\subsection{Real grid data type (rgrid)}
This data type (rgrid) corresponds to a structure with the following members:
\begin{longtable}{p{.33\textwidth} p{.6\textwidth}}
Member & Description\\
\cline{1-2}
REAL *value & Array holding the real grid point data.\\
char id$[32]$ & String describing the grid (comment).\\
size\_t grid\_len & Number of bytes allocated for value array.\\
INT nx & Number of grid points along the 1st grid index ($x$).\\
INT ny & Number of grid points along the 2nd grid index ($y$).\\
INT nz & Number of grid points along the 3rd grid index ($z$).\\
INT nz2 & This is equal to $2\times(nz / 2 + 1)$. Used for indexing *value.\\
        & (the typical FFT convention of storing R2C/C2R data).\\
REAL step & Grid step length (equal in all directions).\\
REAL x0 & Grid origin $x_0$ (default 0.0).\\
REAL y0 & Grid origin $y_0$ (default 0.0).\\
REAL z0 & Grid origin $z_0$ (default 0.0).\\
REAL kx0 & Grid origin $k_{x,0}$ in the reciprocal space (default 0.0).\\
REAL ky0 & Grid origin $k_{y,0}$ in the reciprocal space (default 0.0).\\
REAL kz0 & Grid origin $k_{z,0}$ in the reciprocal space (default 0.0).\\
REAL (*value\_outside) & Pointer to function returning values outside\\
\phantom{X}(rgrid *, INT, INT, INT) & the grid.\\
void *outside\_params\_ptr & Pointer for passing additional data to value\_outside function.\\
REAL default\_outside\_params & Default REAL value for outside\_params\_ptr.\\
fftwX\_plan plan & FFTW plan for forward FFT transforming the grid (X = f, empty, l).\\
fftwX\_plan iplan & FFTW plan for inverse FFT transforming the grid (X = f, empty, l).\\
cufftHandle cufft\_handle\_r2c & CUFFT handle for forward FFT transform (CUDA).\\
cufftHandle cufft\_handle\_c2r & CUFFT handle for inverse FFT transform (CUDA).\\
REAL fft\_norm & Normalization factor for FFT (without including the step length).\\
REAL fft\_norm2 & Normalization factor for FFT (including the step length).\\
char flag & Field for marking whether the grid is exclusively in use or not (rgrid\_claim() and rgrid\_release()).\\
\end{longtable}

\noindent
Requesting rgrid data type is the same as using struct rgrid\_struct. To reference the value at index (i, j, k) in grid-$>$value, use:
\begin{verbatim}
grid->value[(i * grid->ny + j) * grid->nz2 + k]
\end{verbatim}
The same indexing applies to 1-D (i = j = 0) and 2-D grids with (i = 0). This does not account for the grid boundary condition. Library function rgrid\_value\_at\_index(rgrid *grid, INT i, INT j, INT k) can be used to retrieve the value subject to the chosen boundary condition automatically. To set value for a grid point at (i, j, k), library function rgrid\_value\_to\_index(rgrid *grid, INT i, INT j, INT k, REAL value) can be used. This function does not consider the boundary condition assigned to the grid. Note that a different indexing scheme must be used if the grid is in the reciprocal (Fourier) space:
\begin{verbatim}
grid->value[(i * grid->ny + j) * (grid->nz / 2 + 1) + k]
\end{verbatim}
Alternatively, function rgrid\_cvalue\_at\_index(rgrid *grid, INT i, INT j, INT k) can be used.

\noindent
\textbf{Especially when CUDA is used, user applications should NOT access the grids directly but use the routines described above.} They take care of the proper communication with the GPU.

\subsection{Complex grid data type (cgrid)}

Data type cgrid corresponds to a structure with the following members:
\begin{longtable}{p{.43\textwidth} p{.55\textwidth}}
Member & Description\\
\cline{1-2}
REAL complex *value & Array holding the complex grid point data.\\
char id[32] & String describing the grid (comment).\\ 
size\_t grid\_len; & Number of bytes allocated for the value array.\\
INT nx & Number of grid points along the 1st grid index ($x$).\\
INT ny & Number of grid points along the 2nd grid index ($y$).\\
INT nz & Number of grid points along the 3rd grid index ($z$).\\
REAL step & Grid step length (equal in all directions).\\
REAL x0 & Grid origin $x_0$ (default 0.0).\\
REAL y0 & Grid origin $y_0$ (default 0.0).\\
REAL z0 & Grid origin $z_0$ (default 0.0).\\
REAL kx0 & Grid origin $k_{x,0}$ in the reciprocal (momentum) space (default 0.0).\\
REAL ky0 & Grid origin $k_{y,0}$ in the reciprocal (momentum) space (default 0.0).\\
REAL kz0 & Grid origin $k_{z,0}$ in the reciprocal (momentum) space (default 0.0).\\
REAL omega & Rotation frequency around $z$-axis (rotating flow).\\
REAL complex (*value\_outside) & Pointer to function returning values outside\\
\phantom{X}(cgrid *grid, INT, INT, INT) & the grid.\\
void *outside\_params\_ptr & Pointer for passing additional data to value\_outside function.\\
REAL complex default\_outside\_params & Default REAL complex value for outside\_params\_ptr.\\
fftwX\_plan plan & FFTW plan for forward FFT transforming the grid (X = f, empty, l).\\
fftwX\_plan iplan & FFTW plan for inverse FFT transforming the grid (X = f, empty, l).\\
cufftHandle cufft\_handle & CUFFT handle for FFT transforms using CUDA.\\
char host\_lock & If $\ne$ 0 the grid is locked into host memory (no GPU use).\\
REAL fft\_norm & Normalization factor for FFT (without including the step length).\\
REAL fft\_norm2 & Normalization factor for FFT (including the step length).\\
char flag & Field for marking whether the grid is exclusively in use or not (cgrid\_claim() and cgrid\_release()).\\
\end{longtable}
\noindent
Indexing of the grid values follow the same convention as for rgrid data type. See also the note about accessing the grid values directly there.

\subsection{Wave function data type (wf)}

Wave functions are special structures that contain a complex grid (wave function values) and all other necessary parameters such that it can be propagated in time according to a non-linear Schr\"odinger equation.

\begin{longtable}{p{.3\textwidth} p{.65\textwidth}}
Member & Description\\
\cline{1-2}
cgrid *grid & Complex grid containing the wave function values.\\
REAL mass & Particle mass that is represented by this wave function.\\
REAL norm & Requested normalization of the wave function.\\
char boundary & Boundary condition to be used for time propagation. WF\_DIRICHLET\_BOUNDARY = Dirichlet boundary condition, WF\_NEUMANN\_BOUNDARY = Neumann boundary condition, WF\_PERIODIC\_BOUNDARY = Periodic boundary condition.\\
char propagator & Time propagator. WF\_2ND\_ORDER\_FFT = FFT propagator with 2nd order accuracy in time, WF\_4TH\_ORDER\_FFT = FFT propagator with 4th order accuracy in time, WF\_2ND\_ORDER\_CN = Crank-Nicolson propagator with 2nd order accuracy in time, or WF\_4TH\_ORDER\_CN = Crank-Nicolson propagator with 4th order accuracy in time.\\
cgrid *workspace & Workspace required for the chosen propagator (may be NULL if not used).\\
cgrid *workspace2 & Workspace required for the chosen propagator (may be NULL if not used).\\
cgrid *workspace3 & Workspace required for the chosen propagator (may be NULL if not used).\\
REAL (*ts\_func)(INT, INT, INT, void *) & User specified time step function. Allows, for example, for spatially dependent time step length.\\
INT lx, hx; & If non-zero, these indices specify the position of the absorbing boundary along $x$.\\
INT ly, hy; & If non-zero, these indices specify the position of the absorbing boundary along $y$.\\
INT lz, hz;  & If non-zero, these indices specify the position of the absorbing boundary along $z$.\\
\end{longtable}

\chapter{Library functions}

Functions in libgrid are divided into following three classes: 1) rgrid\_* (functions for real valued grids; rgrid), 2) cgrid\_* (functions for complex valued grids; cgrid), and 3) grid\_wf\_* (functions for wavefunctions; wf). These are described in the following subsections. The arguments are listed in the tables in the same order as they are passed to the corresponding functions.

\section{Real grid routines}

\input{real-functions}

\section{Complex grid routines}

\input{complex-functions}

\section{Mixed grid routines}

\input{mixed-functions}

\section{Wavefunction routines}

\input{wf-functions}

\section{Random number routines}

\input{random-functions}

\section{Thread routines}

\input{thread-functions}

\section{FFTW related routines}

\input{fftw-functions}

\section{Linear algebra routines}

\input{linalg-functions}

\section{Interpolation routines}

\input{interpolate-functions}

\section{Timer routines}

\input{timer-functions}

\section{CUDA specific routines}

\input{cuda-functions}

\section{External CUDA routines}

Users may wish to provide their own functions that operate on grids, which causes a problem for CUDA execution. CUDA wants the functions to be compiled separately using nvcc compiler, which would require libgrid users to consider CUDA coding \& compilation. To work around this inconvenience, libgrid allows users to define their on functions (libgrid/src/ext-cuda directory). These are compiled at the time of libgrid compilation, which also makes them accessible in CUDA. Currently, there are several functions defined using this external mechanism, which are used by libdft library. See libgrid/src/ext-cuda directory for examples.

\input{cuda-ext-functions}

\chapter{Interfacing with other software}

\section{Utilities}

Several utilities are available for converting the libgrid binary grid format to other formats that can be read by 3rd party visualization programs (e.g., visit, paraview, xmgrace, veusz). These programs are located in libgrid/util directory. They have to be compiled and installed separately (i.e., make and make install in that directory). The default installation directory is /usr/local/bin. All NETCDF files created will have x, y, and z fields that describe the grid dimensions.

\subsection{grd2cdf -- Convert libgrid format to NETCDF}

To convert test.grd (created by libgrid) to NETCDF format, use:
\begin{verbatim}
% grd2cdf test.grd test.cdf
\end{verbatim}

\subsection{vector2cdf -- Convert scalar and vector field to NETCDF}

Convert one scalar and one vector field to one NETCDF file. For example:
\begin{verbatim}
% vector2cdf density.grd flux_x.grd flux_y.grd flux_z.grd cdffile.cdf
\end{verbatim}
where density.grd is a scalar grid and flux\_x, flux\_y, flux\_z correspond to $x$, $y$, $z$ components of the field. All these will be separately accessible in the NETCDF file and they can be overlaid in graphics presentation. To visualize a vector field in visit, it must first be converted (vecfield is the vector field name):
\begin{enumerate}
\item Select Controls $\rightarrow$ Expressions
\item Enter name: vecfield
\item Type: Vector Mesh Variable
\item Standard Editor: $\lbrace$fx,fy,fz$\rbrace$
\item Press Apply.
\end{enumerate}

\subsection{wf2cdf -- Convert libgrid binary wave function to NETCDF (scalar \& vector field)}

Convert wave function grid to NETCDF file containing: density ($|\psi|^2$) and the three probability flux components: flux\_x, flux\_y, flux\_z. This file can be visualized with visit (see above) or with paraview. In the latter case make sure that the NETCDF file has the extension ``.nc" (rather than ``.cdf"). The fields in NETCDF file are: rho = density, fx = flux\_x, fy = flux\_y, and fz = flux\_z. For example, to convert wave function grid file, use:
\begin{verbatim}
% wf2cdf wavefunction.grd wavefunction.nc
\end{verbatim}

\subsection{wf2cdf2 -- Convert libgrid binary wave function to NETCDF (real \& imaginary parts)}

Similar to wf2cdf but only place real and imaginary parts of the wave function into the NETCDF file (field names re and im). 

\subsection{gview and gview2 -- Animate 1-D data using xmgrace program}

Both gview and gview2 programs will take an ordered list of files as arguments, which will be displayed in that order. There is a short pause at the first file to allow adjustment of the display. View works for ASCII files of 3-D cuts along x, y, and z axes (as output by rgrid\_write\_grid()). View2 works for 1-D grids (e.g., dimension 1, 1, 256). To animate a set of files bubble-*.z (1-D), use:
\begin{verbatim}
% gview2 bubble-{?,??,???,????,?????}.z
\end{verbatim}
where the files were named bubble-1.z, bubble-2.z, etc. The wildcard above (csh/tcsh syntax; bash is likely different) ensures that the files will appear in the correct order. View2 also allows saving the individual images during view. To get this behavior, add -s option. Programs like ffmpeg can be used to create movies from the resulting jpg files.

% TODO: include these from examples directory rather than writing them here.
\chapter{Examples}

\section{Fourier transform example}

\begin{verbatim}
/*
 * Example: FFT of real grid.
 *
 */

#include <stdlib.h>
#include <stdio.h>
#include <math.h>
#include <grid/grid.h>
#include <omp.h>

/* Grid dimensions */
#define NX 256
#define NY 256
#define NZ 256

/* Spatial step length of the grid */
#define STEP 0.5

/* wave vectors top be mapped onto grid */
#define KX 1.0
#define KY 1.0
#define KZ 1.0

/* Function returning standing wave in x, y, and z directions */
REAL func(void *NA, REAL x, REAL y, REAL z) {

  return COS(x * 2.0 * M_PI * KX / (NX * STEP)) 
        + COS(y * 2.0 * M_PI * KY / (NY * STEP)) 
        + COS(z * 2.0 * M_PI * KZ / (NZ * STEP));
}

int main(int argc, char **argv) {

  rgrid *grid;           /* Pointer real grid structure */

  grid_threads_init(0);  /* Use all available cores */

  /* If libgrid was compiled with CUDA support, enable CUDA */
#ifdef USE_CUDA
  cuda_enable(1);
#endif

  /* Allocate real grid with dimensions NX, NY, NZ and spatial step size STEP */
  /* Periodic boundary condition is assigned to the grid */
  grid = rgrid_alloc(NX, NY, NZ, STEP, RGRID_PERIODIC_BOUNDARY, NULL, "test grid");

  /* Map the standing wave function onto the grid */
  rgrid_map(grid, &func, NULL);

  /* Output the grid before FFT */
  rgrid_write_grid("before", grid);

  /* Perform FFT */
  rgrid_fft(grid);

  /* Perform normalize inverse FFT */
  rgrid_inverse_fft_norm(grid);

  /* Write grid after forward & inverse FFTs (we should get the original grid) */
  rgrid_write_grid("after", grid);

  /* If CUDA in use, output usage statistics */
#ifdef USE_CUDA
  cuda_statistics(1);
#endif

  return 0;
}
\end{verbatim}

\section{Spherical average example}

\begin{verbatim}
/*
 * Example: Calculate spherical average of a real grid.
 *
 */
#include <stdlib.h>
#include <stdio.h>
#include <math.h>
#include <grid/grid.h>
#include <omp.h>

/* Grid dimensions */
#define NX 256
#define NY 256
#define NZ 256

/* Spatial step length of the grid */
#define STEP 0.5

/* Binning info */
#define BINSTEP 0.5
#define NBINS (NX / 2)
/* 0 = Calculate spherical average, 1 = Include multiplication by 4pi r^2 */
#define VOLEL 0   

/* Function to be mapped onto the grid */
REAL func(void *NA, REAL x, REAL y, REAL z) {

  return x * x + y * y + z * z;
}

int main(int argc, char **argv) {

  rgrid *grid;
  REAL *bins;
  FILE *fp;
  INT i;
 
  /* Initialize threads (0 = use all threads available) */
  grid_threads_init(0);

  /* If libgrid was compiled with CUDA support, enable CUDA */
#ifdef USE_CUDA
  cuda_enable(1);
#endif
  
  /* Allocate real grid */
  grid = rgrid_alloc(NX, NY, NZ, STEP, RGRID_PERIODIC_BOUNDARY, NULL, "test grid");

  /* Map function func() onto the grid */
  rgrid_map(grid, &func, NULL);

  /* Write the data on disk before starting */
  rgrid_write_grid("before", grid);

  /* Allocate memory for the bins */
  bins = (REAL *) malloc(sizeof(REAL) * NBINS);
  /* Perform spherical average of the grid */
  rgrid_spherical_average(grid, NULL, NULL, bins, BINSTEP, NBINS, VOLEL);

  /* Write spherical average to disk */
  if(!(fp = fopen("after.dat", "w"))) {
    fprintf(stderr, "Can't open file for writing.\n");
    exit(1);
  }
  for (i = 0; i < NBINS; i++)
    fprintf(fp, FMT_R " " FMT_R "\n", BINSTEP * (REAL) i, bins[i]);
  fclose(fp);

  return 0;
}
\end{verbatim}

\section{Poisson equation example}

\begin{verbatim}
/*
 * Example: Solve Poisson equation.
 *
 */

#include <stdio.h>
#include <math.h>
#include <grid/grid.h>

/* Grid dimensions */
#define NX 256
#define NY 256
#define NZ 256
#define STEP 0.2

/* Right hand side function for Poisson equation (Gaussian) */
REAL gaussian(void *arg, REAL x, REAL y, REAL z) {

  REAL inv_width = 0.2;
  REAL norm = 0.5 * M_2_SQRTPI * inv_width;

  norm = norm * norm * norm;
  return norm * exp(-(x * x + y * y + z * z) * inv_width * inv_width);
}

int main(int argc, char **argv) {
  
  rgrid *grid, *grid2;
  
  /* Initialize with 16 OpenMP threads */
  grid_threads_init(16);

  /* If libgrid was compiled with CUDA support, enable CUDA */
#ifdef USE_CUDA
  cuda_enable(1);
#endif
  
  /* Allocate real grid for the right hand side (and the solution) */
  grid = rgrid_alloc(NX, NY, NZ, STEP, RGRID_PERIODIC_BOUNDARY, NULL, "Poisson1");
  grid2 = rgrid_alloc(NX, NY, NZ, STEP, RGRID_PERIODIC_BOUNDARY, NULL, "Poisson2");

  /* Map the right hand side to the grid */
  rgrid_map(grid, gaussian, NULL);

  /* Write right hand side grid */
  rgrid_write_grid("input", grid);

  /* Solve the Poisson equation (result written over the right hand side in grid) */
  rgrid_poisson(grid);  

  /* Write output file (solution) */
  rgrid_write_grid("output", grid);

  /* Check by taking Laplacian (should be equal to input) & write */
  rgrid_fd_laplace(grid, grid2);
  rgrid_write_grid("check", grid2);

  return 0;
}
\end{verbatim}

\section{Wave packet example}

\begin{verbatim}
/*
 * Example: Propagate wavepacket in harmonic potential (3D).
 *
 * Try for example:
 * ./wavepacket 0 128 0.01 200 0.0 0.0 0.0 0.25 0.25 0.25 -2.0 0.0 0.0
 *
 * Although this is 3-D calculation, the above settings will initiate
 * the motion along the x-axis. Therefore it can be visualized by:
 *
 * gview2 output-{?,??,???}.x
 * 
 * (the wildcard syntax is likely different for bash; the above is tcsh)
 *
 * Arguments:
 * 1st  = Number of threads to be used.
 * 2nd  = Number of points in X, Y, Z directions (N x N x N grid).
 * 3rd  = Time step (atomic units).
 * 4th  = Number of iterations to run.
 * 5th  = Initial wave vector (momentum) along X (atomic units).
 * 6th  = Initial wave vector (momentum) along Y (atomic units).
 * 7th  = Initial wave vector (momentum) along Z (atomic units).
 * 8th  = Initial wave packet width in X direction (atomic units).
 * 9th  = Initial wave packet width in Y direction (atomic units).
 * 10th = Initial wave packet width in Z direction (atomic units).
 * 11th = Initial wave packet center along X direction (atomic units).
 * 12th = Initial wave packet center along Y direction (atomic units).
 * 13th = Initial wave packet center along Z direction (atomic units).
 *
 */

#include <stdlib.h>
#include <stdio.h>
#include <math.h>
#include <grid/grid.h>
#include <omp.h>

/* Define this for 4th order accuracy in time */
/* Otherwise for 2nd order accuracy */
#define FOURTH_ORDER_FFT

REAL complex wavepacket(void *arg, REAL x, REAL y, REAL z);
REAL complex harmonic(void *arg, REAL x, REAL y, REAL z);

/* Wave packet structure */
typedef struct wparams_struct {
  REAL kx, ky, kz;
  REAL wx, wy, wz;
  REAL xc, yc, zc;
} wparams;

/* Harmonic potential parameters */
typedef struct pparams_struct {
  REAL kx, ky, kz;
} pparams;

int main(int argc, char *argv[]) {

  INT l, n, iterations, threads;
  REAL step, lx, time_step;
  REAL complex time;
  wf *gwf = NULL;
  cgrid *potential;
  rgrid *rworkspace;
  char fname[256];
  pparams potential_params;
  wparams wp_params;
  
  /* Parameter check */
  if (argc != 14) {
    fprintf(stderr, "Usage: wavepacket <thr> <npts> <tstep> <iters> <kx> <ky> <kz> <wx> <wy> <wz> <xc> <yc> <zc>\n");
    return -1;
  }
  
  /* Parse command line arguments */
  threads = atoi(argv[1]);
  n = atoi(argv[2]);
  time_step = atof(argv[3]);
  iterations = atol(argv[4]);
  wp_params.kx = atof(argv[5]);
  wp_params.ky = atof(argv[6]);
  wp_params.kz = atof(argv[7]);
  wp_params.wx = atof(argv[8]);
  wp_params.wy = atof(argv[9]);
  wp_params.wz = atof(argv[10]);
  wp_params.xc = atof(argv[11]);
  wp_params.yc = atof(argv[12]);
  wp_params.zc = atof(argv[13]);

  if(wp_params.wx == 0.0 || wp_params.wy == 0.0 || wp_params.wz == 0.0) {
    fprintf(stderr, "Width cannot be zero.\n");
    exit(1);
  }
  
  /* Set spatial grid step length based on number of grid points */
  step = 0.4 / (((REAL) n) / 16.0);
  
  fprintf(stderr, "Grid (" FMT_I "X" FMT_I "X" FMT_I ")\n", n, n, n);
  
  /* Potential parameters */
  lx = ((REAL) n) * step;
  /* Force constants for the harmonic potential */
  potential_params.kx = lx * 2.0;
  potential_params.ky = lx * 2.0;
  potential_params.kz = lx * 2.0;
  
  /* Initialize OpenMP threads */
  grid_threads_init(threads);
  
  /* allocate memory (mass = 1.0) */
  gwf = grid_wf_alloc(n, n, n, step, 1.0, WF_PERIODIC_BOUNDARY, 
#ifdef FOURTH_ORDER_FFT
                      WF_4TH_ORDER_FFT, "WF");
#else
                      WF_2ND_ORDER_FFT, "WF");
#endif
  potential = cgrid_alloc(n, n, n, step, CGRID_PERIODIC_BOUNDARY, 0, "potential");
  rworkspace = rgrid_alloc(n, n, n, step, RGRID_PERIODIC_BOUNDARY, 0, "rworkspace");
  
  /* Initialize wave function */
  grid_wf_map(gwf, wavepacket, &wp_params);
  grid_wf_normalize(gwf);
  
  /* Map potential */
  cgrid_smooth_map(potential, harmonic, &potential_params, 1);
  
  /* Propagate */
  time = time_step;
  for(l = 0; l < iterations; l++) {
    printf("Iteration " FMT_I " with wf norm = " FMT_R "\n", l, grid_wf_norm(gwf));
    /* Write |psi|^2 to output-* files */
    grid_wf_density(gwf, rworkspace);
    sprintf(fname, "output-" FMT_I, l);
    rgrid_write_grid(fname, rworkspace);
    /* Propagate one time step */
    grid_wf_propagate(gwf, potential, time);
  }

  /* Release resources */
  grid_wf_free(gwf);
  rgrid_free(rworkspace);
  cgrid_free(potential);
  
  return 0;
}

/* Function for creating the initial wave packet */
REAL complex wavepacket(void *arg, REAL x, REAL y, REAL z) {

  REAL kx = ((wparams *) arg)->kx;
  REAL ky = ((wparams *) arg)->ky;
  REAL kz = ((wparams *) arg)->kz;
  REAL wx = ((wparams *) arg)->wx;
  REAL wy = ((wparams *) arg)->wy;
  REAL wz = ((wparams *) arg)->wz;
  REAL xc = ((wparams *) arg)->xc;
  REAL yc = ((wparams *) arg)->yc;
  REAL zc = ((wparams *) arg)->zc;
  REAL x2, y2, z2;

  x -= xc;
  y -= yc;
  z -= zc;
  x2 = x / wx; x2 *= x2;
  y2 = y / wy; y2 *= y2;
  z2 = z / wz; z2 *= z2;

  return CEXP(- x2 + I * kx * x - y2  + I * ky * y - z2 + I * kz * z);
}

/* Function for harmonic potential */
REAL complex harmonic(void *arg, REAL x, REAL y, REAL z) {

  pparams params = *((pparams *) arg);

  return 0.5 * (params.kx * params.kx * x * x + params.ky * params.ky * y * y 
                + params.kz * params.kz * z * z);
}
\end{verbatim}

\end{document}
